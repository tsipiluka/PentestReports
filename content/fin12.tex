\section{Finding 12 - Weak Cipher Suites for Webserver on Port 443}
%center under chapter title a one row table with 6 coloumns and no borders
\vspace*{-0,3cm}
\begin{center}
    \begin{tabular}{c c c c}
        \textbf{Classification:} & Weak Cryptography & \textbf{Severity:} & \textbf{\textcolor{red}{High}}  
        \end{tabular}
\end{center}

Performing an nmap scan on the port 443 of the \ac{DUT} reaveals that th webserver is using weak cipher suites.

\subsection*{Finding Impact}
Weak cipher suites are vulnerable to attacks like the SWEET32 attack. This allows an attacker to read the data which is transmitted between the client and the webserver.

\subsection*{Finding Details}
Following nmap command was executed on the \ac{DUT}:
\begin{lstlisting}[language=bash]
$ nmap -sV --script ssl-enum-ciphers -p 443 172.16.0.29
\end{lstlisting}
 The following output shows the weak cipher suites which are used by the webserver:
\begin{lstlisting}[language=bash]
|   64-bit block cipher 3DES vulnerable to SWEET32 attack
|   64-bit block cipher DES vulnerable to SWEET32 attack
|   64-bit block cipher DES40 vulnerable to SWEET32 attack
|   64-bit block cipher IDEA vulnerable to SWEET32 attack
|   64-bit block cipher RC2 vulnerable to SWEET32 attack
|   Broken cipher RC4 is deprecated by RFC 7465
|   Ciphersuite uses MD5 for message integrity
|   Export key exchange
|   Insecure certificate signature (SHA1), score capped at F
\end{lstlisting}

\subsection*{Evaluation of Results}
\begin{center}
    \begin{tabular}{cccc}
    \textbf{Effort to Fix:} & &\ \textbf{\textcolor{orange}{Medium}}\
    \end{tabular}
\end{center}

The webserver should only use strong cipher suites. This can be done by updating the configuration of the webserver.
