\section{Finding 4 - Vulnerable Apache Version}
%center under chapter title a one row table with 6 coloumns and no borders
\vspace*{-0,3cm}
\begin{center}
    \begin{tabular}{c c c c}
        \textbf{Classification:} & Vulnerable Software Version & \textbf{Severity:} & \textbf{\textcolor{orange}{Medium}}  
        \end{tabular}
\end{center}

%make spacing on top of the table less
\vspace*{-0,8cm}
\begin{center}
    \begin{tabular}{c c}
        \textbf{CVE:} & CVE-2023-25690, CVE-2023-27522, CVE-2006-20001, \\ & CVE-2022-36760, CVE-2022-37436
    \end{tabular}
\end{center}

On port 80 the \ac{DUT} is running a vulnerable Apache version (\textbf{”Apache 2.4.54”}). This version has multiple vulnerabilities and shouldn't be used in production. \newline
The following vulnerabilities are known from \ac{CVE} but haven't been exploited on the \ac{DUT}. Some of these vulnerabilities may only be exploitable with specific configurations. Nevertheless, all of these vulnerabilities are shown to provide transparency and to show the possible impact of the vulnerabilities.


\subsection{Finding Impact}
\textbf{CVE-2023-25690:} When the mod\_proxy configuration is enabled a HHTP smuggling attack is possible, which could bypass the access controls.
\newline\newline
\textbf{CVE-2023-27522:} This vulnerability allows an attacker to send a origin header which contains special characters to the server. This could be used truncate/split the response forwarded to the client.
\newline\newline
\textbf{CVE-2006-20001:} This vulnerability allows an attacker to send a specific if request to the server, which could be used to crash the process.
\newline\newline
\textbf{CVE-2022-36760:} Due to an incosistent interpretation of HTTP requests of the server it could be possible for attackers to smuggle HTTP requests to the \ac{AJP} server. 
\newline\newline
\textbf{CVE-2022-37436:} A malicious backend has the ability to terminate the response headers prematurely, leading to certain headers being integrated into the response body. Following headers which serve a security function, they will not be comprehended by the client.

\subsection{Finding Details}
%add code snippet
\begin{lstlisting}[language=bash]
$ nmap -A 172.16.0.29

Starting Nmap 7.93 ( https: //nmap.org ) at 2023-03-06 09:30 CET
Nmap scan report for 172.16.0.29
Host is up (0.00051s latency).

PORT STATE SERVICE VERSION
80/tep open http Apache httpd 2.4.54 ((Debian))
\end{lstlisting}

\subsection{Evaluation of Results}
\begin{center}
    \begin{tabular}{cccc}
    \textbf{Effort to Fix:} & &\ \textbf{\textcolor{green!50!blue}{Low}}\
    \end{tabular}
\end{center}
To fix this vulnerability the Apache Server has to be updated to a newer version. This could be done with the following command:
\begin{lstlisting}[language=bash]
$ apt update && apt install apache2
\end{lstlisting}
