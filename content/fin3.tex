\section{Finding 4 - Vulnerable Apache Version}
%center under chapter title a one row table with 6 coloumns and no borders
\vspace*{-0,3cm}
\begin{center}
    \begin{tabular}{c c c c}
        \textbf{Classification:} & Vulnerable Software Version & \textbf{Severity:} & \textbf{\textcolor{orange}{Medium}}  
        \end{tabular}
\end{center}

%make spacing on top of the table less
\vspace*{-0,8cm}
\begin{center}
    \begin{tabular}{c c}
        \textbf{CVE:} & CVE-2023-25690, CVE-2023-27522, CVE-2006-20001, \\ & CVE-2022-36760, CVE-2022-37436
    \end{tabular}
\end{center}

On port 80 the \ac{DUT} is running a vulnerable Apache version (\textbf{”Apache 2.4.54”}). This version has multiple vulnerabilities and shouldn't be used in production.


\subsection{Finding Impact}
\textbf{CVE-2023-25690:} When the mod\_proxy configuration is enabled a HHTP smuggling attack is possible.
\newline\newline
\textbf{CVE-2023-27522:} This vulnerability allows an attacker to send a origin header which contains special characters to the server. This could be used runcate/split the response forwarded to the client.
\newline\newline
\textbf{CVE-2006-20001:} This vulnerability allows an attacker to send an if request 
\subsection{Finding Cause}

\subsection{Finding Details}

\subsection{Evaluation of Results}
\begin{center}
    \begin{tabular}{cccc}
    \textbf{Effort to Fix:} & &\ \textbf{\textcolor{orange}{Medium}}\
    \end{tabular}
\end{center}
How do you judge the individual technical findings (severity, likelihood)?
What is your suggested remediation, if there is one?
How can the customer validate their remediation is effective once implemented?