\section{Finding 1 - Exact SSH-Version can be determined}
%center under chapter title a one row table with 6 coloumns and no borders
\begin{center}
\begin{tabular}{cccccccc}
\textbf{Classification:} & Information Disclosure & \textbf{CVE:} & & \textbf{Severity:} & Low &\\
\end{tabular}
\end{center}


\subsection{Finding Description}
A nmap port scan reveils the exact version of the running SSH-Server on the \ac{DUT}.
The version used on the \ac{DUT} is \textbf{”OpenSSH 8.4p1 Debian 5+deb11u1”} and can be accessed via port 22.


\subsection{Finding Impact}
This information can be used by an attacker to find known vulnerabilities in this specific SSH-Version to exploit the \ac{DUT}.

\subsection{Finding Cause}
This finding is caused by SSH itself. There is no configuration option to hide the version of the SSH-Server. The version-banner can be found in the sshd binary.
\subsection{Finding Details}
\begin{lstlisting}[language=bash]
$ nmap -sV -p 22
Starting Nmap 7.91 ( https://nmap.org ) at 2021-06-01 14:00 CEST
Nmap scan report for
Host is up (0.00020s latency).
PORT STATE SERVICE VERSION
22/tcp open ssh OpenSSH 8.4p1 Debian 5+deb11u1 (protocol 2.0)
\end{lstlisting}

\subsection{Evaluation of Results}
    \begin{center}
    \begin{tabular}{cccc}
    \textbf{Effort to Fix:} & &\\
    \end{tabular}
    \end{center}

How do you judge the individual technical findings (severity, likelihood)?
What is your suggested remediation, if there is one?
How can the customer validate their remediation is effective once implemented?