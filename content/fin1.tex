\section{Finding 1 - Exact OpenSSH-Version can be determined}
%center under chapter title a one row table with 6 coloumns and no borders
\begin{center}
\begin{tabular}{cccccccc}
\textbf{Classification:} & Information Disclosure & \textbf{CVE:} & & \textbf{Severity:} & \textbf{\textcolor{green!50!blue}{Low}} &\\
\end{tabular}
\end{center}


\subsection*{Finding Description}
A nmap port scan reveils the exact version of the running OpenSSH-Server on the \ac{DUT}.
The version used on the \ac{DUT} is \textbf{”OpenSSH 8.4p1 Debian 5+deb11u1”} and can be accessed via port 22.


\subsection*{Finding Impact}
This information can be used by an attacker to find known vulnerabilities in this specific OpenSSH-Version to exploit the \ac{DUT}. Possible exploitations can be found in Finding 2.

\subsection*{Finding Cause}
This finding is caused by OpenSSH itself. There is no configuration option to hide the version of the SSH-Server. The version-banner can be found in the sshd binary. 
\newline
\subsection*{Finding Details}
\begin{lstlisting}[language=bash]
$ nmap -A 172.16.0.29
Starting Nmap 7.91 ( https://nmap.org ) at 2023-03-06 09:30 CEST
Nmap scan report for 172.16.0.29
Host is up (0.00051s latency).
PORT STATE SERVICE VERSION
22/tcp open ssh OpenSSH 8.4p1 Debian 5+deb11u1 (protocol 2.0)
\end{lstlisting}

\subsection*{Evaluation of Results}
    \begin{center}
    \begin{tabular}{cccc}
    \textbf{Effort to Fix:} & &\ \textbf{\textcolor{orange}{Medium}}\
    \end{tabular}
    \end{center}
To fix this finding the OpenSSH Binary has to be changed. By default the binary can be found at '/usr/sbin/sshd'. Change the binary with hexedit and search for the version banner. After removing the version banner restart the ssh service wit 'systemctl restart sshd.service'.
Due to the fact of the risk  of working on the binary itself, this finding is rated as medium effort to fix.
\newline

