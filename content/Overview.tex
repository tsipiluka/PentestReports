\chapter{Introduction}

\section{Scope}
This Penetration Test Report is based on the E-Mail from our client Pr. Dr. Bauer. The E-Mail was send on 2023-02-20 13:56 with the subject ”Schriftliche Beschreibung der Laborarbeit 'Offensive Security'” (SHA-224 sum: \newline  dafaf185c6d7ec66804121fd25b8f1165f96aea3e183efbb660d250d). The given scenario is a black box test.
The given \ac{DUT} is a Raspberry Pi.
The \ac{DUT} might be interacting with external systems. Those systems are not included in the scope of this test.
The \ac{DUT} is running on a ”Cortex-A53” CPU which is based on an aarch64 ARM architecture (whole CPU information can be found in the appendix). The running OS is Debian with a 5.15.61-v8+ kernel (whole kernel information can be found in the appendix).



\newpage
\section{Severities}
Each finding in this report is assigned a severity level. The following table defines the severity levels used in this report. Some findings may be estimated different in the organizational context.

\begin{table}[H]
    \centering
    \label{tab:severity}
    \begin{tabular}{|c|p{10cm}|}
    \hline
    \multicolumn{1}{|c|}{\textbf{Severity Level}} & \multicolumn{1}{c|}{\textbf{Definition}} \\ \hline
    \textcolor{green!50!blue}{\textbf{Low}} & Vulnerability that has a limited impact on the system or data and may not require immediate attention. It represents a low risk to the organization and can be addressed in a routine patching cycle or by implementing a simple configuration change.\\ \hline
    \textcolor{orange}{\textbf{Medium}} & Vulnerability that has a moderate impact on the system or data and requires some effort to exploit. It represents a moderate risk to the organization and may require a more thorough analysis and remediation effort. \\ \hline
    \textcolor{red}{\textbf{High}} & Vulnerability that has a significant impact on the system or data and can be easily exploited. It represents a high risk to the organization and requires immediate attention and remediation. \\ \hline
    \end{tabular}
    \caption{Severity Levels}
\end{table}
    
    
    

\section{Classification}
Each finding in this report is assigned to a classification. The following table defines the classification levels used in this report. Notice that some findings could be assigned to multiple classifications. For a better overview in this report every finding is assigned only to one classification.

\begin{table}[H]
    \centering
    \label{tab:classification}
    \begin{tabular}{|c|p{10cm}|}
    \hline
    \multicolumn{1}{|c|}{\textbf{Classification}} & \multicolumn{1}{c|}{\textbf{Definition}} \\ \hline
    \textbf{Information Disclosure} & Information disclosure vulnerabilities are those that allow an attacker to obtain sensitive information from the system.\\ \hline
    \textbf{Denial of Service} & Denial of service vulnerabilities are those that allow an attacker to prevent the system from providing its services.\\ \hline
    \textbf{Elevation of Privilege} & Elevation of privilege vulnerabilities are those that allow an attacker to gain access to resources that are normally protected from the user.\\ \hline
    \textbf{Misonfiguration} & Vulnerabilities that are caused by configurations and can lead to an exploit. \\ \hline
    \end{tabular}
    \caption{Classification}
    \end{table}
\section{Effort to Fix}
Each finding in this report is assigned to an effort to fix level. The following table defines the effort to fix levels used in this report. Notice that this is only a recommendation. Some findings may be estimated different in the organizational context.

\begin{table}[H]
    \centering
    \label{tab:effort}
    \begin{tabular}{|c|p{10cm}|}
    \hline
    \multicolumn{1}{|c|}{\textbf{Effort to Fix Level}} & \multicolumn{1}{c|}{\textbf{Definition}} \\ \hline
    \textcolor{green!50!blue}{\textbf{Low}} & The vulnerability can be fixed with a simple configuration change or a routine patching cycle.\\ \hline
    \textcolor{orange}{\textbf{Medium}} & The vulnerability can be fixed with a moderate effort for example with a different or new implementation.\\ \hline
    \textcolor{red}{\textbf{High}} & The vulnerability can be fixed with a high effort like an architectural change.\\ \hline
    \end{tabular}
    \caption{Effort to Fix}
\end{table}
