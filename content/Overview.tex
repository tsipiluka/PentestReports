\chapter{Introduction CHECKS NOCH ENTFERNEN}

\section{Scope}
This Penetration Test Report is based on the E-Mail from our client Pr. Dr. Bauer. The E-Mail was send on 2023-02-20 13:56 with the subject ”Schriftliche Beschreibung der Laborarbeit 'Offensive Security'” (HASHSUMME EINFÜGEN). The given scenario is a black box test.
The given \ac{DUT} is a Raspberry Pi GENAUE DATEN EINSETZEN.
The os is LINUX VERSION (uname) EINFÜGEN.
The \ac{DUT} might be interacting with external systems. Those systems are not in the scope of this test.



\section{Severities}
For each vulnerability you uncover in your testing, you typically provide:
The likelihood of exploitation, taking into account how easy it is to discover and exploit
The impact on the control you get once exploited
Suggested remediation
Suggested validation of remediation effectiveness

Low
Moderate
High
Severe
Critical

\begin{table}[h]
    \centering
    \label{tab:severity}
    \begin{tabular}{|c|p{10cm}|}
    \hline
    \multicolumn{1}{|c|}{\textbf{Severity Level}} & \multicolumn{1}{c|}{\textbf{Definition}} \\ \hline
    \textcolor{green!50!blue}{\textbf{Low}} & Vulnerability that has a limited impact on the system or data and may not require immediate attention. It represents a low risk to the organization and can be addressed in a routine patching cycle or by implementing a simple configuration change.\\ \hline
    \textcolor{orange}{\textbf{Medium}} & Vulnerability that has a moderate impact on the system or data and requires some effort to exploit. It represents a moderate risk to the organization and may require a more thorough analysis and remediation effort. \\ \hline
    \textcolor{red}{\textbf{High}} & Vulnerability that has a significant impact on the system or data and can be easily exploited. It represents a high risk to the organization and requires immediate attention and remediation. \\ \hline
    \end{tabular}
    \caption{Severity Levels}
    \end{table}
    
    
    

\section{Classification}

\section{Effort to Fix}
Low
Moderate
High
