\section{Finding 19 - Vulnerable OpenSSL Version}
%center under chapter title a one row table with 6 coloumns and no borders
\vspace*{-0,3cm}
\begin{center}
    \begin{tabular}{c c c c}
        \textbf{Classification:} & Vulnerable Software Version & \textbf{Severity:} & \textbf{\textcolor{red}{High}}  
        \end{tabular}
\end{center}
\vspace*{-0,8cm}
\begin{center}
    \begin{tabular}{c c c}
        \textbf{CVE:} & CVE-2014-0076 & CVE-2014-0160
    \end{tabular}
\end{center}

Executing the following command on the \ac{DUT} shows the installed version of OpenSSL:
\begin{lstlisting}
openssl version
\end{lstlisting}
This shows that the \ac{DUT} is running a vulnerable version of OpenSSL:\newline \textbf{OpenSSL 1.0.1b} 

\subsection*{Finding Impact}
The mentioned OpenSSL version is vulnerable to the following CVEs: \newline \newline
\textbf{CVE-2014-0076:} An error exists related to the implementation of the Elliptic Curve Digital Signature Algorithm (ECDSA) that could allow nonce disclosure via the 'FLUSH+RELOAD' cache side-channel attack. 
\newline \newline
\textbf{CVE-2014-0160:} An out-of-bounds read error, known as the 'Heartbleed Bug', exists related to handling TLS heartbeat extensions that could allow an attacker to obtain sensitive information such as primary key material, secondary key material and other protected content.



\subsection*{Evaluation of Results}
\begin{center}
    \begin{tabular}{cccc}
    \textbf{Effort to Fix:} & &\ \textbf{\textcolor{green!50!blue}{Low}}\
    \end{tabular}
\end{center}
Update OpenSSL with the following command:
\begin{lstlisting}
    sudo apt-get install openssl
\end{lstlisting}