\section{Finding 7 - No Brute-Force Protection for SSH}

%center under chapter title a one row table with 6 coloumns and no borders
\vspace*{-0,3cm}
\begin{center}
    \begin{tabular}{c c c c}
        \textbf{Classification:} & Misconfiguration & \textbf{Severity:} & \textbf{\textcolor{red}{High}}  
        \end{tabular}
\end{center}

\vspace*{-0,8cm}
\begin{center}
    \begin{tabular}{c c}
        \textbf{CVE:} & 
    \end{tabular}
\end{center}

As seen in Finding 6 the password of the user ”bluey” can be brute-forced. Even though the weak password is a finding on its own, there should be also a protection against brute-force attacks. This could have stopped the attack in Finding 6. 

\subsection*{Evaluation of Results}
\begin{center}
    \begin{tabular}{cccc}
    \textbf{Effort to Fix:} & &\ \textbf{\textcolor{green!50!blue}{Low}}\
    \end{tabular}
\end{center}
To protect against brute-force attacks the following configuration should be updated/added to the sshd\_config file:
\begin{lstlisting}[language=bash]
MaxTries 3
\end{lstlisting}
Also a multifactor authentication could be used for the ssh service.