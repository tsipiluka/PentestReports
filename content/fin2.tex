\clearpage
\section{Finding 2 - Vulnerable OpenSSH Version}
%center under chapter title a one row table with 6 coloumns and no borders
\begin{center}
\begin{tabular}{c c c c}
    \textbf{Classification:} & Vulnerable Software Version & \textbf{Severity: \textcolor{orange}{Medium}} &  \\ \textbf{CVE:} & CVE-2021-28041, CVE-2021-41617 &\\
\end{tabular}
\end{center}


\subsection*{Finding Description}
The \ac{DUT} is running a vulnerable OpenSSH version (8.4p1). This version is vulnerable to the following CVEs: CVE-2021-28041, CVE-2021-41617.

\subsection*{Finding Impact}
Following exploits can be used to gain access to the \ac{DUT}:
\newline
\newline
\textbf{CVE-2021-28041:} This vulnerability enables an attacker to carry out unauthorized code execution on a target system remotely. The vulnerability stems from an error in the ssh-agent, where a remote attacker can lure the victim to connect to a server where the attacker has root access.
\newline
\newline
\textbf{CVE-2021-41617:} When OpenSSH is used with non default configurations privilige escalation is possible. (Check configuration)

\subsection*{Evaluation of Results}
\begin{center}
    \begin{tabular}{cccc}
    \textbf{Effort to Fix:} & &\ \textbf{\textcolor{green!50!blue}{Low}}\
    \end{tabular}
\end{center}
Update to newer OpenSSH version. This can be done by running the following command:
\begin{lstlisting}[language=bash]
$ sudo apt update
$ sudo apt install openssh-server
\end{lstlisting}

